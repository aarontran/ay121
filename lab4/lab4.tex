\documentclass[10pt]{article}

\usepackage[margin=1in, letterpaper]{geometry}
\usepackage{parskip}

\usepackage{amsthm, amsmath, amssymb}
\usepackage{gensymb}  % For use of degree symbol
\usepackage[pdftex]{graphicx}
\usepackage{hyperref}

\usepackage{enumerate} % For use of (a), (b), et cetera
\usepackage{booktabs} % Tables
\usepackage[margin=20pt, labelfont=bf, labelsep=period,
justification=justified]{caption} % Captions in figure floats

% ======================
% Document setup, layout
% ======================

% The following metadata will show up in the PDF properties
\hypersetup{
	colorlinks = true,
	urlcolor = magenta,  % Links to URLs
	linkcolor = blue,  % Links within PDF
	pdfauthor = {Aaron Tran},
	pdfkeywords = {berkeley},
	pdftitle = {Astro 121, UG Radio Lab, Lab 4 - \today},
	pdfsubject = {},
	pdfpagemode = UseNone
}

% Don't indent paragraphs
\setlength\parindent{0em}

% Slightly more compact lines
\linespread{0.95}

% ===============
% Useful commands
% ===============
\newcommand {\mt}{\mathrm}
\newcommand {\unit}[1]{\; \mt{#1}}
% http://vemod.net/typesetting-units-in-latex

% Sets, operators
\newcommand {\ints}{\mathbb{Z}}
\newcommand {\ptl}{\partial}
\newcommand {\dl}{\nabla}

\begin{document}

% =======
% Titling
% =======
\begin{center}
\Large{Partial $1420\unit{MHz}$ HI Survey of the North Polar Spur}

\normalsize
\textbf{Aaron Tran}${}^{1,3}$ \\
Isaac A. Domagalski${}^{1,2}$, Caleb Levy${}^{1,2}$ \\
Aaron Parsons${}^{2,4,5}$, Garrett K. Keating${}^{2,4,5}$, Baylee Bordwell${}^{2,5}$ \\
\footnotesize
${}^1$Central Intelligence Agency, 1000 Colonial Farm Rd, McLean, VA 22101, USA \\
${}^2$Dept. Astronomy, UC Berkeley, D-23 Hearst Field Annex, Berkeley, CA 94720, USA \\
${}^3$Dept. Earth and Planetary Science, UC Berkeley, 335 McCone Hall, Berkeley, CA 94720, USA \\
${}^4$Radio Astronomy Laboratory, UC Berkeley, Berkeley, CA 94720, USA \\
${}^5$Undergraduate Radio Laboratory teaching staff \\
\textit{Submitted 2014 May ??}
\end{center}

% ========
% Abstract
% ========
\section*{Abstract}

We observe the north polar spur and stuff

% ============
% Introduction
% ============
\section{Introduction}

background on NPS

% ============
% Observations
% ============
\section{Observations}

% ------------------------
% Leuschner specifications
% ------------------------
\subsection{Leuschner radio dish}

We use the Leuschner radio dish ($37\degree 55' 10.2'' \unit{N}$, $-122\degree 09' 12.4'' \unit{E}$), operated by UC Berkeley as part of Leuschner Observatory, to collect single-dish observations of the hyperfine HI line.  The Leuschner radio dish, hereafter Leuschner (Figure \ref{fig:kartp}), has diameter $3.6\unit{m}$ or $4.5\unit{m}$ depending on who is asked; the beamwidth is $\sim4\degree$ at its operating frequency of $1420 \unit{MHz}$.  Leuschner's view at low altitudes is blocked by surrounding hills; to the north Leuschner may point above $\sim50\degree$, to the south Leuschner may point to $20$--$30\degree$ altitude.  The Leuschner radio dish was built for the SETI Rapid Prototype Array (an early prototype for the now-underfunded and incomplete Allen Telescope Array); the dish has since been appropriated for undergraduate education.

\begin{figure}[!ht]
    \centering
    \includegraphics[width=0.5\textwidth]{kartp.png} \\
    \caption{The Leuschner dish has beamwidth $\sim 4 \degree$ at $1420 \unit{MHz}$.  Here the dish is shown with its erstwhile caretaker, \emph{kartp} (courtesy of I. Domagalski, E. Herrera, K. Moses).}
    \label{fig:kartp}
\end{figure}

RF waves incident on Leuschner are passed through a $200 \unit{MHz}$ bandpass filter centered on $1420 \unit{MHz}$ and mixed with a local oscillator (LO) signal of frequency $f_{\mt{LO}}$; both operations are performed at the antenna feed.  The LO mixing sends frequencies of interest near $1420 \unit{MHz}$ to $\sim150 \unit{MHz}$; this down-converted signal is routed to Leuschner Observatory facilities and bandpass filtered at $145$--$155 \unit{MHz}$.

The signal is digitized by an FPGA-based spectrometer using a polyphase filter bank; the effective sampling rate is $24 \unit{MHz}$ giving a bandwidth $144$--$156 \unit{MHz}$; the signal of interest appears in our frequency output via Nyquist aliasing [Siemion, 2012].  To characterize system temperature and frequency-dependent gain during data reduction, we collect observations at two LO frequencies $f_{\mt{LO}} = 1268.9 \unit{MHz}$ and $f_{\mt{LO}} = 1271.9 \unit{MHz}$.  The bandwidth $144$--$156 \unit{MHz}$ thus corresponds to the radio frequency bands $1412.9$--$1424.9 \unit{MHz}$ and $1415.9$--$1427.9 \unit{MHz}$ respectively.

% Observing campaign
\subsection{Observing campaign}

We observed the region of the sky with galactic latitude $b\geq 0\degree$ and galactic longitudes between $l=210\degree$ and $l=20\degree$, which contains the North Polar Spur.over the timespan of 2014 April 26 to 2014 May 5.

Unfortunately, blah was not visible from the interferometer during our observing campaign and could not be mapped.

Blah was not visible
from the dish and was not mapped.

In order to completely map the sky, the region of interest should be sampled with spacing $2\degree$ (for beamwidth $4\degree$).  In reality, hah.  We sampled most of the available region at $4\degree$ spacing

% ==============
% Data reduction
% ==============
\section{Data reduction}

\subsection{RFI removal}

\subsection{Calibration}


\subsection{Image generation}

I'm afraid to admit that, due to a shortage of time, I did not write my own
velocity computation etc... scripts.  I was able to start on it, but did not
work on baseline fitting and removal and peak identification.  Working on it
now would require branching Isaac's pipeline, and risk messing up the current
pipeline.  So I shan't do that, FOR NOW.

Later on, I would really want to be able to decompose the various peaks.  It
looks like our spur observations don't have multiple peaks anyways, so it's not
so bad since we dgaf about the galactic plane.

% =======
% Results
% =======
\section{Results}

\begin{figure}[!ht]
    \centering
    \includegraphics[width=0.5\textwidth]{plots/col_density.png} \\
    \caption{Column density}
    \label{fig:fuck1}
\end{figure}
\begin{figure}[!ht]
    \centering
    \includegraphics[width=0.48\textwidth]{plots/veloc_mean.png}
    \includegraphics[width=0.48\textwidth]{plots/veloc_std.png} \\
    \caption{Mean velocity (left), stdev of velocity (right)}
    \label{fig:fuck2}
\end{figure}

To dos...

(manual Mollweide projection + colorbars + zooming + better plots is partway done...)

completely overhaul images (manually compute mollweide projection and image interpolation instead of relying on meshgrid)
include colorbars
determine physiologically sensible colormaps
determine best nonlinear colormapping
reduce pdf file sizes...

% ----------------------
% Blah
% ----------------------
\subsection{Blah}

% ==========
% Discussion
% ==========
\section{Discussion}

% ===========
% Conclusions
% ===========
\section{Conclusions}

\section{Acknowledgments}

\begin{figure}[!ht]
    \centering
    \includegraphics[width=0.5\textwidth]{kartp2.png} \\
    \caption{(image courtesy of I. Domagalski, E. Herrera, K. Moses)}
    \label{fig:kartp2}
\end{figure}

\begin{center}
Kartp noster, qui es in radiolab:\\
sanctificetur nomen tuum;\\
adveniat regnum tuum;\\
fiat voluntas tua.\\
sicut in academia, et in universitas.\\
Observationem nostrum cotidianum da nobis hodie;\\
et dimitte nobis errores nostra,\\
sicut et nos dimittimus erroribus nostris;\\
et ne nos inducas in tentationem;\\
sed libera nos a circumsonum.
\end{center}

%Karto and Baylee are gr8.
%Isaac and Caleb are gr8.
%Aaron Parsons is cool.

\section{Author contributions}

I. A. D. did stuff.  C. L. did stuff.  A. T. did stuff.

\section{Electronic supplement}

All supporting files are stored on the repository:\\
\href{https://github.com/aarontran/ay121}
{https://github.com/aarontran/ay121/lab4/}.

\section{References}

\hangindent 0.25in Condon, J. J. and S. M. Ransom (2006), Essential Radio Astronomy, \\
\href{http://www.cv.nrao.edu/course/astr534/ERA.shtml}
{http://www.cv.nrao.edu/course/astr534/ERA.shtml}.

\hangindent 0.25in Green, R. M. (1985), \textit{Spherical astronomy}, 520pp.,
Cambridge Univ. Press, Cambridge.

\hangindent 0.25in Siemion, A. (2012), Leuschner Spectrometer, CASPER documentation wiki, \\
\href{https://casper.berkeley.edu/wiki/Leuschner\_Spectrometer}
{https://casper.berkeley.edu/wiki/Leuschner\_Spectrometer}.

\hangindent 0.25in Wolleben, M. (2007), A New Model for the Loop I (North Polar
Spur) Region, \textit{Astrophys. J.}, \textit{664}, 349--356,
doi:10.1086/518711.

\end{document}
