\documentclass[11pt]{article}

\usepackage[margin=1in, letterpaper]{geometry}
\usepackage{parskip}
\usepackage{amsthm, amsmath, amssymb}
\usepackage{enumerate} % For use of (a), (b), et cetera
%\usepackage[pdftex]{graphicx}
\usepackage{hyperref}

% The following metadata will show up in the PDF properties
\hypersetup{
	colorlinks = true,
	urlcolor = black,
	pdfauthor = {Aaron Tran},
	pdfkeywords = {berkeley},
	pdftitle = {Astro 121, UG Radio Lab, Lab 1 - \today},
%	pdfsubject = {},
	pdfpagemode = UseNone
}

% Don't indent paragraphs
\setlength\parindent{0em}

% Problem numbering
\newcounter{iternum}
\setcounter{iternum}{0}
\newcommand{\prob}{\stepcounter{iternum} \textbf{\arabic{iternum}.} }
\newcommand{\probcont}{\textbf{\arabic{iternum}. (cont.) }}

% ===============
% Useful commands
% ===============
\newcommand {\mt}{\mathrm}
\newcommand {\unit}[1]{\; \mt{#1}}
% http://vemod.net/typesetting-units-in-latex

% ==========================
% Document specific commands
% ==========================

\begin{document}

\section{Abstract}

Give numbers (e.g., transmission line speed)

\section{Introduction}

Don't go that long.  Keep it simple.  Whatever you need for it to feel
complete, more for you.

\section{Methods}

\section{Voltage dividers}

We use two $10 \unit{k\Omega}$ resistors to form a resistive voltage divider.
When applying a DC current, we measure $V_\mt{in} = 6.13 \unit{V}$
and $V_\mt{out} = 3.05 \unit{V}$.  Current is $0.30 \unit{mA}$.
Strangely, when we apply a $1 \unit{MHz}$ signal with $V_\mt{pp} = 5 \unit{V}$,
we find extreme signal attenuation.  This also occurs at $1 \unit{kHz}$.
When we change our resistors to $1.2 \unit{k\Omega}$, we observe
a signal scaled by $1/2$ as expected.  We do not currently understand why
$10 \unit{k\Omega}$ resistors cause such a decreased gain.

We add a second voltage divider to the output of the first, using the
same resistors ($10 \unit{k\Omega}$).  For what we assume is the same
input voltage, $6.13 \unit{V}$, we measure $1.20 \unit{V}$ at the output
of the second voltage divider.

The impedance of a voltage divider is dependent on the application.

The Th\`{e}venin equivalent resistance of the first divider, seen from
its output, is BLAH.  Thevenin voltage is Vth = Vin*R2/(R1+R2).  Thevenin
current is Vin/R1.  Then, Zth is R2*R1/(R1+R2) as expected.

The equivalent resistance of infinitely many parallel resistors is zero.

We use two $10 \unit{nF}$ capacitors to form a capacitive voltage divider.
The capacitors halve sinusoidal input signals at both $1 \unit{kHz}$ and
$1 \unit{MHz}$ as expected.  If we add a resistor to ground at the output,
we expect:
\begin{equation}
  \frac{V_{out}}{V_{in}} = \frac{1}{2 + 1/(j\omega RC)}
\end{equation}
where $j$ is the imaginary unit, $\omega$ is signal angular frequency (rad/s),
and $R$ and $C$ are resistor and capacitor values ($\Omega$, F).  In effect,
the circuit now behaves as a high-pass filter.

\section{RC, LC, and RLC filters}

Text

\section{Diodes}

Text

\section{FM demodulator}

Text

\section{Results}

\section{Discussion}

\section{Conclusion}

What you're gonna do with this next.  Don't rewrite your intro/abstract.

Use plain fucking english.  Don't worry about being technical, or too
concise... otherwise you're fucked.

\end{document}