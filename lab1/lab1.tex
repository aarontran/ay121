\documentclass[11pt]{article}

\usepackage[margin=1in, letterpaper]{geometry}
\usepackage{parskip}
\usepackage{amsthm, amsmath, amssymb}
\usepackage{enumerate} % For use of (a), (b), et cetera
%\usepackage[pdftex]{graphicx}
\usepackage{hyperref}

% The following metadata will show up in the PDF properties
\hypersetup{
    colorlinks = true,
    urlcolor = black,
    pdfauthor = {Aaron Tran},
    pdfkeywords = {berkeley},
    pdftitle = {Astro 121, UG Radio Lab, Lab 1 - \today},
%   pdfsubject = {},
    pdfpagemode = UseNone
}

% Don't indent paragraphs
\setlength\parindent{0em}

% Problem numbering
\newcounter{iternum}
\setcounter{iternum}{0}
\newcommand{\prob}{\stepcounter{iternum} \textbf{\arabic{iternum}.} }
\newcommand{\probcont}{\textbf{\arabic{iternum}. (cont.) }}

% ===============
% Useful commands
% ===============
\newcommand {\mt}{\mathrm}
\newcommand {\unit}[1]{\; \mt{#1}}
% http://vemod.net/typesetting-units-in-latex

% ==========================
% Document specific commands
% ==========================

\begin{document}

\section{Abstract}

Karto's general rule: one sent. for each sect.
Give numbers (e.g., transmission line speed)

\section{Introduction}

Modulation of radio frequency (RF) electromagnetic waves is a simple and
effective method for long distance information transmission, as exemplified by
amplitude/frequency modulated radio.  The generation, transmission, and
reception of modulated signals are interesting engineering problems which are
also accessible to interested amateurs and beginning students of electronics.
Simple modulators and demodulators may be built with a minimum
of passive and active components (i.e., enough to fit on a small breadboard);
transmitting/receiving antennae may be jury rigged with bottles and wire.
RF circuits hold great educational potential for undergraduate students.

Here we present the design and construction of an FM receiver ($1.045
\unit{MHz}$ with bandwidth $200 \unit{kHz}$)

Don't go that long.  Keep it simple.  Whatever you need for it to feel
complete, more for you.  Enough to interpret information presented later.

If length(intro) > length(discussion) you're doing it wrong.

\section{FM receiver design}

* Circuit diagram of receiver, highlighting blocks of components that act to form a function

\subsection{Passive diode FM demodulator}

* Describe overall design of receiver, with subsections for each functional plot.  List salient features for each (-3dB point, bias voltages @ various locations, etc.). Present as if you invented it, and are writing a seminal paper (hahahaha)

* Plot expected output of LC filter in receiver as function of $f$.  Mark transmission band on plot and explain rationale behind its placement on response curve.

* Plot bandpass of final filter that defines band of output audio signal.  Describe (maybe plot) rationale for selecting said filter.

\subsection{Common-emitter amplifier}

* Describe speaker amplifier (circuit diagram), describe how it works, parts, bias voltages.  Define operating band - what frequency range, and what sets the bounds?

* Suppose speaker amplifier connects to speaker at end of long wire.  Given the circuit we built, what impedance cable do you recommend, how to terminate it (with 8-ohm speaker) to avoid reflections?  Don't worry about maximizing current through speaker.

* Specify input/output impedance of circuit at audio frequencies



\section{Results}

* Discuss transmission line results.

* Explain our odd troubles with the diode detector.  Whether the circuit helped was very equivocal...

* Discuss frequency / clipping issues here?

* Noise measurement

\section{Discussion}

* Amplifier test - explain Allan variance test, but give reasonable physical numbers

\section{Conclusion}

What you're gonna do with this next.  Don't rewrite your intro/abstract.

Use plain english.  Don't worry about being technical, or too concise...

\section{Acknowledgments}

Circuit diagrams were generated using Fritzing.

\section{References}

\end{document}
